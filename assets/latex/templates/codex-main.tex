% Phoenix 2.0 Apex Edition - Master Codex Template
% Professional typesetting for ceremonial documentation

\documentclass[12pt,twoside,openright]{book}

% ============================================================
% PACKAGES
% ============================================================

% Font and typography
\usepackage{fontspec}
\usepackage{microtype}          % Better typography
\usepackage[english]{babel}

% Fonts (requires XeLaTeX)
\setmainfont{EB Garamond}
\setsansfont{Cinzel}           % For headers
\setmonofont{Fira Code}        % For code/sigils

% Page layout
\usepackage[
  paperwidth=6in,
  paperheight=9in,
  inner=0.75in,
  outer=0.5in,
  top=0.7in,
  bottom=0.8in,
  includeheadfoot
]{geometry}

% Colors
\usepackage{xcolor}
\definecolor{phoenixred}{RGB}{180,0,0}
\definecolor{hydrogenesiblue}{RGB}{0,60,120}
\definecolor{apexgold}{RGB}{184,134,11}
\definecolor{substratepurple}{RGB}{75,0,130}

% Enhanced boxes and frames
\usepackage{tcolorbox}
\tcbuselibrary{skins,breakable}

% Mathematical notation
\usepackage{amsmath,amssymb,amsthm}

% Graphics and figures
\usepackage{graphicx}
\usepackage{float}

% Tables
\usepackage{booktabs}
\usepackage{longtable}

% Code listings
\usepackage{listings}
\lstset{
  basicstyle=\ttfamily\small,
  breaklines=true,
  frame=single,
  backgroundcolor=\color{gray!10}
}

% Headers and footers
\usepackage{fancyhdr}
\pagestyle{fancy}
\fancyhf{}
\fancyhead[LE]{\sffamily\small\leftmark}
\fancyhead[RO]{\sffamily\small\rightmark}
\fancyfoot[C]{\thepage}
\renewcommand{\headrulewidth}{0.4pt}

% Cross-references
\usepackage{hyperref}
\hypersetup{
  colorlinks=true,
  linkcolor=phoenixred,
  urlcolor=hydrogenesiblue,
  citecolor=apexgold,
  pdftitle={Phoenix 2.0 Apex Edition},
  pdfauthor={Hydrogenesi Foundation}
}

% Index
\usepackage{makeidx}
\makeindex

% Custom spacing
\setlength{\parskip}{6pt}
\setlength{\parindent}{0pt}
\linespread{1.2}

% ============================================================
% CUSTOM COMMANDS
% ============================================================

% Operator symbols (ensure font has these)
\newcommand{\opGenesis}{⊕}
\newcommand{\opHarmonic}{⊗}
\newcommand{\opRecursive}{⊛}
\newcommand{\opApex}{△}
\newcommand{\opVoid}{⊝}
\newcommand{\opMirror}{⊞}
\newcommand{\opConvergence}{⊳}
\newcommand{\opDivergence}{⊲}

% Law box environment
\newtcolorbox{lawbox}[1][]{
  colback=apexgold!5,
  colframe=apexgold!80,
  fonttitle=\sffamily\bfseries,
  title={#1},
  breakable,
  enhanced
}

% Sigil environment (preserves spacing)
\lstnewenvironment{sigilenv}
{
  \lstset{
    basicstyle=\ttfamily\normalsize,
    frame=none,
    backgroundcolor=\color{white},
    xleftmargin=0pt,
    framexleftmargin=0pt,
    breaklines=false,
    columns=fixed
  }
}
{}

% Ritual steps
\newenvironment{ritualsteps}
  {\begin{enumerate}\setlength{\itemsep}{6pt}}
  {\end{enumerate}}

% Chapter styling
\usepackage{titlesec}
\titleformat{\chapter}[display]
  {\normalfont\huge\sffamily\bfseries\color{phoenixred}}
  {\chaptertitlename\ \thechapter}{20pt}{\Huge}
\titlespacing*{\chapter}{0pt}{0pt}{40pt}

% ============================================================
% DOCUMENT METADATA
% ============================================================

\title{
  {\Huge\sffamily\bfseries Phoenix 2.0 Apex Edition}\\[12pt]
  {\Large\sffamily The Unified Codex}\\[6pt]
  {\large\sffamily A Ceremonial Architecture of Recursive Identity}
}
\author{Hydrogenesi Foundation}
\date{2026}

% ============================================================
% DOCUMENT CONTENT
% ============================================================

\begin{document}

% Front Matter
\frontmatter
\maketitle

\chapter*{Preface}
This codex represents the unification of Phoenix and Hydrogenesi engines into a coherent framework of recursive identity. Through three layers—Substrate, Universal, and Apex—we explore the mathematics of emergence, the geometry of triadic balance, and the ceremony of symbolic precision.

\tableofcontents
\listoffigures

% Main Content
\mainmatter

% Include chapter files (these would be separate .tex files)
% \include{../chapters/seven-laws}
% \include{../chapters/substrate}
% \include{../chapters/universal}
% \include{../chapters/apex}
% \include{../chapters/operators}
% \section{Triad Architecture}

The \textbf{Triad Architecture} is the three-dimensional binding structure that connects the Sovereign Kernel, transformation operators, and harmonic resonance into a unified topological framework.

\subsection{Triad Definition}

A Triad $\mathcal{T}$ is a three-component structure:

\begin{equation}
\mathcal{T} = (\mathcal{K}, \mathcal{O}, \mathcal{H})
\end{equation}

where:
\begin{itemize}
    \item $\mathcal{K}$ is the Sovereign Kernel
    \item $\mathcal{O}$ is the Operator Space (all available transformations)
    \item $\mathcal{H}$ is the Harmonic Field (resonance patterns)
\end{itemize}

\subsection{The Three Axes}

The Triad operates across three fundamental axes:

\subsubsection{Sovereignty Axis}

The sovereignty axis spans from complete autonomy to full integration:

\begin{equation}
\mathcal{S}_{\text{axis}}: \text{Autonomy} \leftrightarrow \text{Integration}
\end{equation}

Position along this axis determines the degree of kernel independence versus inter-kernel cooperation.

\subsubsection{Transformation Axis}

The transformation axis spans from genesis to apex:

\begin{equation}
\mathcal{T}_{\text{axis}}: \text{Genesis} \leftrightarrow \text{Apex}
\end{equation}

Movement along this axis represents the progression from initial state to culmination.

\subsubsection{Harmonic Axis}

The harmonic axis spans from resonance to void:

\begin{equation}
\mathcal{H}_{\text{axis}}: \text{Resonance} \leftrightarrow \text{Void}
\end{equation}

This axis governs the intensity and coherence of harmonic patterns.

\subsection{Triad Topology}

The Triad forms a three-dimensional topological space with specific properties:

\subsubsection{Metric Space}

The Triad space is a metric space with distance function:

\begin{equation}
d(\mathcal{T}_1, \mathcal{T}_2) = \sqrt{\alpha \|K_1 - K_2\|^2 + \beta \|O_1 - O_2\|^2 + \gamma \|H_1 - H_2\|^2}
\end{equation}

where $\alpha, \beta, \gamma$ are weighting coefficients for each component.

\subsubsection{Binding Manifold}

The valid binding states form a manifold $\mathcal{M}$ in the Triad space:

\begin{equation}
\mathcal{M} = \{(\mathcal{K}, \mathcal{O}, \mathcal{H}) \mid \text{Coherent}(\mathcal{K}, \mathcal{O}, \mathcal{H})\}
\end{equation}

where $\text{Coherent}$ is a predicate ensuring valid binding configurations.

\subsubsection{Apex Attractor}

Each Triad has an apex attractor point $\mathcal{A}$ toward which it evolves:

\begin{equation}
\lim_{t \to \infty} \mathcal{T}(t) \to \mathcal{A}
\end{equation}

under repeated application of harmonic operators.

\subsection{Tension Mechanics}

Tension $\tau$ is the driving force in the Triad system:

\begin{equation}
\tau = \nabla U(\mathcal{T})
\end{equation}

where $U$ is the potential energy function of the current Triad state.

\subsubsection{Tension Types}

\begin{itemize}
    \item \textbf{Constructive Tension}: Drives toward apex ($\tau > 0$)
    \item \textbf{Destructive Tension}: Induces regression ($\tau < 0$)
    \item \textbf{Neutral Tension}: Maintains equilibrium ($\tau = 0$)
\end{itemize}

\subsection{Binding Protocol}

The binding protocol ensures coherent interaction between Triad components:

\begin{equation}
\text{Bind}(\mathcal{T}_1, \mathcal{T}_2) \rightarrow \mathcal{T}_{\text{composite}}
\end{equation}

The composite Triad inherits properties from both source Triads while maintaining coherence.

\subsubsection{Binding Constraints}

For a valid binding:

\begin{align}
|\mathcal{K}_1 \cap \mathcal{K}_2| &= \emptyset \quad \text{(kernel disjointness)} \\
\mathcal{O}_{\text{composite}} &= \mathcal{O}_1 \cup \mathcal{O}_2 \quad \text{(operator union)} \\
\mathcal{H}_{\text{composite}} &= \mathcal{H}_1 \otimes \mathcal{H}_2 \quad \text{(harmonic product)}
\end{align}

\subsection{Triad Transformations}

Transformations in the Triad space follow operator application rules:

\begin{equation}
\mathcal{T}' = \text{Op}(\mathcal{T}) = (\mathcal{K}, \text{Op}(\mathcal{O}), \text{Resonate}(\mathcal{H}, \text{Op}))
\end{equation}

The kernel remains invariant while operators modify the operator space and harmonic field.

\subsection{Apex Formation}

An apex state $\mathcal{T}_{\text{apex}}$ is characterized by:

\begin{align}
\text{Complexity}(\mathcal{T}_{\text{apex}}) &= \max \\
\text{Coherence}(\mathcal{T}_{\text{apex}}) &= 1 \\
\text{Stability}(\mathcal{T}_{\text{apex}}) &= \max
\end{align}

Apex formation represents the culmination of recursive transformations.

% \include{../chapters/phoenix}
% \include{../chapters/hydrogenesi}
% \include{../chapters/sigils}
% \include{../chapters/rituals}

% Example chapter content
\chapter{The Seven Universal Laws}

\begin{lawbox}[Law of Recursive Identity]
Every entity contains the pattern of its own generation.

\textbf{Formal Expression:} $\forall x: x = \opRecursive(x_0)$

\begin{sigilenv}
    ⊛
   ╱│╲
  ⊛ ⊛ ⊛
 ╱│╲│╱│╲
⊛ ⊛ ⊛ ⊛ ⊛
\end{sigilenv}
\end{lawbox}

Identity is not static but emerges through self-referential iteration. Each element carries within it the algorithmic trace of its creation, enabling reconstruction and transformation.

% Back Matter
\backmatter

% \include{../appendices/glossary}
% \include{../appendices/operators}

\printindex

\end{document}
